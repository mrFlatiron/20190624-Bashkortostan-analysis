\begin{frame} 
\frametitle{Макроэкономический анализ. Цели работы}

\begin{enumerate}
	\item Определить динамику ВРП.
	\item Оценить уровень безработицы в регионе, а так же конкретные отрасли с недостатком или переизбытком кадров.
	\item Собрать статистику по размерам оплаты труда в различных отраслях экономики.
	\item Сравнить уровни цен на различные категории товаров с другими регионами.
	\item Оценить демографическую ситуацию в регионе.
\end{enumerate}

\end{frame}

\begin{frame} 
\frametitle{Основной источник}

\begin{center}
\Large{БашкортостанСтан: \myref{http://bashstat.gks.ru/}{http://bashstat.gks.ru/}}
\end{center}

\end{frame}

\begin{frame} 
\frametitle{Распределение ВРП по отраслям}

\insertpicc{slava/vrp-2017.png}{Динамика ВРП на душу населения в рублях}

\end{frame}

\begin{frame} 
\frametitle{Распределение ВРП по отраслям}

Отрасли, доля ВРП которых составляет меньше 1 процента:
\begin{itemize}
	\item Водоснабжение, водоотведение, сбор отходов, ликвидация загрязнений
	\item Деятельность финансовая и страховая
	\item Деятельность в области культуры, спорта и развлечений
\end{itemize}

Доля остальных отраслей, включая образование, научную и техническую деятельность и здравоохранение составляет от 1 до 5 процентов.

\end{frame}

\begin{frame}
\frametitle{Динамика ВРП}
\insertpicc{slava/vrp-dynamics.png}{Динамика ВРП на душу населения в рублях}
\end{frame}

\begin{frame}
\frametitle{Факты}
\begin{itemize}
	\item Уровень безработицы на 2017 г. составил 5.6\%.
	\item Средняя заработная плата в январе 2019 года составила 30357,7.
	\item Средняя заработная плата по России в 2018 составила 43400
\end{itemize}
\end{frame}

\begin{frame}
\frametitle{Востребованность профессий. Нехватка кадров.}
Рабочие профессии, требующие квалификации и специальных навыков. Например:\\ автоэлектрик, машинист экскаватора, монтажник, оператор станков с программным управлением, сварщик, слесарь, токарь, электрик и т. д.
\end{frame}

\begin{frame}
\frametitle{Востребованность профессий. Переизбыток кадров.}
Должности служащих. Например:\\
администратор, бухгалтер, делопроизводитель, инспектор по кадрам, консультант, секретарь, социальный работник, \textbf{экономист}, юрисконсульт.
\end{frame}

\begin{frame}
\frametitle{Демография. Естественное движение}
\begin{center}
\begin{tabular}{|c|c|c|}
	\hline
	& Человек & На 1000 чел. \\
	\hline
	Родилось & 49315 & 12.1 \\
	\hline
	Умерло &  50387 & 12.4 \\
	\hline
	Естественный прирост (убыль) & -1072 & -0.3\\
	\hline
\end{tabular}
\end{center}
\end{frame}

\begin{frame}
\frametitle{Демография. Миграция}
\begin{center}
\begin{tabular}{|c|c|}
	\hline
	& Человек  \\
	\hline
	Прибыло &  143762 \\
	\hline
	Убыло &  146369 \\
	\hline
	Прирост (убыль) & -2607 \\
	\hline
\end{tabular}
\end{center}
\end{frame}