\section{Макроэкономический анализ}

Обозначим цели данного макроэкономического анализа Башкортостана:
\begin{enumerate}
	\item Определить динамику ВРП.
	\item Оценить уровень безработицы в регионе, а так же конкретные отрасли с недостатком или переизбытком кадров.
	\item Собрать статистику по размерам оплаты труда в различных отраслях экономики.
	\item Сравнить уровни цен на различные категории товаров с другими регионами.
	\item Оценить демографическую ситуацию в регионе.
\end{enumerate}

\subsection{Валовой региональный продукт}
\textit{Валовой региональный продукт} - показатель, измеряющий валовую добавленную стоимость, исчисляемый путём исключения из суммарной валовой продукции объёмов её промежуточного потребления. На национальном уровне ВРП соответствует валовому национальному продукту, который является одним из базовых показателей системы национальных счетов.

Однако сумма ВРП по всем субъектам государства не равна ВВП, так как  такая сумма не включает добавленную стоимость по нерыночным коллективным услугам (оборона, государственное управление и так далее), оказываемым государственными учреждениями обществу в целом.

Данные БашкортостанСтата публикуются в открытом доступе на сайте \href{http://bashstat.gks.ru/wps/wcm/connect/rosstat_ts/bashstat/ru/statistics/}{\textcolor{blue}{Федеральной Службы Государственной Статистики}}. Информация о ВРП, опубликованная 29.04.2019,  содержит информацию только за 2017 и 2016 год с пометкой, что данные за 2017 год предварительные. Справедливости ради нужно заметить, что та же информация по г. Москва, опубликованная в марте 2019-го отражает ВРП за 2015 и 2016 год.

Отрасли с крупнейшими долями ВРП указаны на следующей круговой диаграмме:
\insertpicc{slava/vrp-2017.png}{Статьи ВРП  за 2017г.}

Отрасли, доля ВРП которых составляет меньше 1 процента:
\begin{itemize}
	\item Водоснабжение, водоотведение, сбор отходов, ликвидация загрязнений
	\item Деятельность финансовая и страховая
	\item Деятельность в области культуры, спорта и развлечений
\end{itemize}

Доля остальных отраслей, включая образование, научную и техническую деятельность и здравоохранение составляет от 1 до 5 процентов.

Для анализа динамики ВРП воспользуемся другой статистикой - показателем ВРП на душу населения с 2010 по 2016 г.
\insertpicc{slava/vrp-dynamics.png}{Динамика ВРП на душу населения в рублях}

\subsection{Рынок труда}
По данным на 2017 год имеется следующая статистика по численности рабочей силы, а так же не входящим в неё людей в возрасте от 15 до 72 лет (тысяч человек):

\begin{tabular}{|c|c|c|c|}
	\hline
	& Мужчины & Женщины & Всего \\
	\hline
	Численность рабочей силы & 1051,7 & 941,1 &  1992,8 \\
	\hline
	Занятые & 998.3 & 883,0 & 1881,3 \\
	\hline
	Безработные$ ^*$ & 53,4 & 58,1 & 111,5 \\
	\hline
	Лица, не входящие в состав РС & 375,3 & 631,3 & 1006,6 \\
	\hline
	Получают пособие & 7,8 & 11,8 & 19,6\\
	\hline
\end{tabular}

*-обращались в службу занятости.

\fact{Таким образом уровень безработицы на 2017 г. составляет 5.6\%.}

Далее приведем краткую сводку средней зарплаты по отраслям. На данной диаграмме представлены пять самых низкооплачиваемых  и пять самых высокооплачиваемых отраслей.

\insertpicc{slava/avg-salary-2019}{Средние зарплаты}

\fact{Средняя заработная плата в январе 2019 года составила 30357,7.}

Средняя заработная плата по России в 2018 составила 43400 согласно \myref{https://bit.ly/2EhQN10}{Википедии}.

\href{http://www.bashzan.ru/posts/120111}{\textcolor{blue}{Информационный портал}} занятости населения 
Министерства семьи и труда РБ предоставляет агрегированную информацию о состоянии рынка труда за январь-март 2019:

\smallskip

\begin{tabular}{|c|c|}
\hline
Количество вакансий & 70000 \\
\hline
В поисках работы в ЦЗН обратилось людей & 1150 \\
\hline
Средняя продолжительность «существования» вакансий & 1,6 месяца \\
\hline
\end{tabular}

\bigskip

\textbf{Наиболее востребованы на рынке труда:}

\textbf{Рабочие профессии:} автоэлектрик, бетонщик, водитель автомобиля, грузчик, дворник, каменщик, кондуктор, кровельщик, курьер, кухонный рабочий, маляр, машинист бульдозера, машинист крана, машинист экскаватора, монтажник, монтажник по монтажу стальных железобетонных конструкций, облицовщик-плиточник, овощевод, оператор станков с программным управлением, отделочник, официант, охранник, парикмахер, плотник, подсобный рабочий, проводник пассажирского вагона, рабочий, сварщик, слесарь по ремонту автомобилей, слесарь-ремонтник, слесарь-сантехник, станочник широкого профиля, токарь, тракторист, фрезеровщик, швея, шлифовщик, штукатур-маляр, электрик, электрогазосварщик, электромонтажник, электромонтер по ремонту и обслуживанию электрооборудования, электросварщик ручной сварки.

\textbf{Должности служащих:} агент страховой, агроном, воспитатель, врач, врач-педиатр, врач-терапевт, инженер-конструктор, инспектор, медицинская сестра (медицинский брат), преподаватель, фельдшер, полицейский.

\bigskip

\textbf{Предложение рабочей силы значительно превышает спрос:}

\textbf{Рабочие профессии:} бармен, гардеробщик, дояр, кладовщик, оператор котельной, \textbf{оператор ЭВМ}, помощник воспитателя, почтальон, продавец, санитар (санитарка), сторож, уборщик производственных и служебных помещений.

\textbf{Должности служащих:} администратор, бухгалтер, главный бухгалтер, делопроизводитель, заведующий складом, заведующий хозяйством, инспектор по кадрам, консультант, лаборант, секретарь, социальный работник, старший бухгалтер, техник, \textbf{экономист}, юрисконсульт.

\subsection{Уровень цен}
Согласно Мосгорстату и Башкортостанстату средние цены к весне 2019 года следующие (руб): следующие (руб.)\\

\begin{tabular}{|c|c|c|}
	\hline
	Продукт & Москва & Башкортостан \\
	\hline
	Молоко пастеризованное 2,5-3,2\% & 72.45 & 45.34 \\
	\hline
	Бензин марки АИ-95 & 46.31 & 44.34 \\
	\hline
	Кв. метр недвижимости$ ^*$ &  225684 & 72130\\
	\hline
\end{tabular}

*- согласно \myref{https://www.irn.ru/rating/moscow/}{ИРН} по району Раменки и \myref{https://www.domofond.ru/tseny-na-nedvizhimost/bashkortostan/ufa-c1514}{domofond.ru} в среднем по Уфе

\bigskip

\begin{tabular}{|c|c|c|}
	\hline
	& Москва & Башкортостан \\
	\hline
	Средняя з/п & 79680 & 30357,7\\
	\hline

\end{tabular}
\subsection{Демографическая ситуация}
Самые свежие данные по естественному движению населения и миграции опубликованы на Башкортостанстате за 2017 год. 
\newpage
Естественное движение:

\begin{tabular}{|c|c|c|}
	\hline
	& Человек & На 1000 человек населения \\
	\hline
	Родилось & 49315 & 12.1 \\
	\hline
	Умерло &  50387 & 12.4 \\
	\hline
	Естественный прирост (убыль) & -1072 & -0.3\\
	\hline
\end{tabular}

\bigskip

Миграция:

\begin{tabular}{|c|c|}
	\hline
	& Человек  \\
	\hline
	Прибыло &  143762 \\
	\hline
	Убыло &  146369 \\
	\hline
	Прирост (убыль) & -2607 \\
	\hline
\end{tabular}

\bigskip

В среднем по России за 2017 естественное движение:

\begin{tabular}{|c|c|c|}
	\hline
	& Человек & На 1000 человек населения \\
	\hline
	Родилось & 1689884 & 11.5 \\
	\hline
	Умерло &  1824340 & 12.42 \\
	\hline
	Естественный прирост (убыль) & -134456 & -0.9\\
	\hline
\end{tabular}

\bigskip

Таким образом демографическая ситуация в Башкортостане лучше, чем в среднем по России.
Основной причиной убыли населения Башкортостана является миграция.
