\section{Анализ проблем рынка труда}

Выпишем очевидные проблемы рынка труда Башкортостана.

Средняя зарплата в Башкортостане (с учетом колебаний этой характеристики за период с 2018 по 2019 год) примерно на 30\% ниже средней по России. Также БашкортостанСтат не предоставляет статистики по медианной з/п, что является более важным показателем, чем средняя. Недостаток квалифицированных кадров рабочих профессий в таких значимых отраслях экономики региона, как строительство и обрабатывающая промышленность является серьезной проблемой на фоне переизбытка специалистов финансового сектора.

Низкий престиж рабочих профессий обусловлен неудовлетворительными условиями труда и смутными перспективами карьерного роста. Неудовлетворительные условия могут включать в себя 
\begin{enumerate}
	\item Низкую з/п
	\item Задержку выплат з/п
	\item Отсутствие роста з/п
	\item Неоплачиваемые принудительные переработки
	\item Отсутствие здоровой корпоративной культуры
	\item Отсутствие бонусов в виде медицинской страховки или путёвок в оздоровительные или иные курортные учреждения
	\item Некачественные инструменты и оборудование, предоставляемые рабочим
\end{enumerate}

Место работы с такими проблемами не сможет удержать молодого специалиста, даже если тот пришел устраиваться под влиянием агрессивной пропаганды в СМИ. Учитывая высокие доли ВРП строительства и обрабатывающей промышленности, региону важно защитить представителей рабочих профессий там, где нарушаются их права начальниками предприятий. 
Предлагается
\begin{enumerate}
	\item Создать реестр представителей рабочих профессий (ПРП)
	\item Создать рабочую группу по защите прав ПРП, обязанную ежегодно собирать и анализировать отзывы о местах работы ПРП, посредством заполнения онлайн анкет с подтверждением личности с гарантией конфиденциальности.
	\item Создать реестр недобросовестных работодателей с публичным доступом, заполнение которого осуществляет рабочая группа.
\end{enumerate}

Такие меры призваны простимулировать развитие конкурентных условий труда промышленников, а также повысить осведомленность ПРП об альтернативных рабочих местах.