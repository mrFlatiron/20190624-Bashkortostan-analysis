\section{Органы власти}

Республика Башкортостан, согласно Конституции республики, является демократическим правовым государством в составе Российской Федерации, выражающим волю и интересы всего многонационального народа республики.

Государственная власть в Республике Башкортостан осуществляется на основе разделения на законодательную, исполнительную и судебную.

Государственную власть в Республике Башкортостан осуществляют Государственное Собрание - Курултай Республики Башкортостан, Глава Республики Башкортостан, Правительство Республики Башкортостан, местные органы государственной власти Республики Башкортостан, суды Республики Башкортостан.

Высшим и единственным законодательным (представительным) органом государственной власти Республики Башкортостан является Государственное Собрание - Курултай Республики Башкортостан. Государственное Собрание Республики Башкортостан состоит из 120 депутатов и избирается сроком на пять лет.

Главой и высшим должностным лицом Республики Башкортостан является Глава Республики Башкортостан.

Глава Республики Башкортостан представляет Республику Башкортостан в отношениях с Президентом Российской Федерации, Советом Федерации и Государственной Думой Федерального Собрания Российской Федерации, Правительством Российской Федерации, иными федеральными органами государственной власти, органами государственной власти субъектов Российской Федерации, органами государственной власти Республики Башкортостан, органами местного самоуправления, общественными объединениями, другими организациями и должностными лицами и при осуществлении международных и внешнеэкономических связей.

Срок полномочий Главы Республики Башкортостан составляет пять лет.

Высшим исполнительным органом государственной власти Республики Башкортостан является Правительство Республики Башкортостан, возглавляемое Главой Республики Башкортостан.

Судебную власть в Республике Башкортостан осуществляют Конституционный суд РБ, федеральные суды и мировые судьи.

В Республике Башкортостан признается и гарантируется местное самоуправление, обеспечивающее самостоятельное решение населением вопросов местного значения, владение, пользование и распоряжение муниципальной собственностью. Структура органов местного самоуправления определяется населением самостоятельно.

Столицей Республики Башкортостан является город Уфа.

Республика входит в состав Приволжского федерального округа Российской Федерации.

\subsection{Структура власти}

Правительство  Республики  Башкортостан  возглавляет Президент
Республики Башкортостан.
     Высшим    исполнительным    органом   государственной   власти
Республики    Башкортостан    является   Правительство   Республики
Башкортостан.

\insertpicc{emil/vlast.jpg}{Органы власти реуспублики Башкортостан}

Министерство  Республики Башкортостан является республиканским
органом  исполнительной  власти,  осуществляющим  в  пределах своей
компетенции   отраслевое  управление  процессами  экономического  и
социального развития.
     
     
Государственный   комитет   Республики  Башкортостан  является
республиканским  органом  исполнительной  власти, осуществляющим на
коллегиальной   основе   межотраслевую   координацию  по  вопросам,
отнесенным  к  его  ведению, а также функциональное регулирование в
определенной сфере деятельности.
    
Ведомство   Республики   Башкортостан   (комитет,  управление,
комиссия    и    инспекция)    является   республиканским   органом
исполнительной   власти,   осуществляющим   в  установленной  сфере
деятельности   и   в   пределах   своей   компетенции   специальные
исполнительные, контрольные, разрешительные или надзорные функции.
     
     
Правительство     Республики     Башкортостан     координирует
деятельность   министерств,   государственных  комитетов  и  других
органов исполнительной власти Республики Башкортостан.

\insertpicc{emil/adminka.jpg}{Дом республики}

Республиканские   органы  исполнительной  власти  осуществляют
свою  деятельность на основе и во исполнение Конституции Российской
Федерации,     федеральных    законов,    Конституции    Республики
Башкортостан,  настоящего Закона и иных нормативных правовых актов,
принятых   Российской   Федерацией  и  Республикой  Башкортостан  в
соответствии с их компетенцией.

Республиканские   органы   исполнительной   власти   в   своей
деятельности     руководствуются     принципами,     установленными
Конституцией   Российской   Федерации,   федеральными   законами  и
Конституцией Республики Башкортостан.

Республиканские  органы  исполнительной  власти обеспечиваются
необходимым   для   их   деятельности   имуществом   -  зданиями  и
сооружениями,  материально-техническими  средствами, оборудованием,
организуют  учет  материальных  и  денежных  средств и осуществляют
контроль за их использованием.

Решения    о   закреплении   имущества   за   соответствующими
республиканскими   органами   исполнительной   власти   принимаются
Правительством Республики Башкортостан.

\subsection{Парламент}

Государственное Собрание – Курултай Республики Башкортостан является высшим и единственным законодательным (представительным) органом государственной власти, состоящим из 110 депутатов.

Депутатом может стать гражданин Российской Федерации, достигший на день голосования 21 года и обладающий избирательным правом. Выборы проводятся на основе всеобщего равного и прямого избирательного права при тайном голосовании. Порядок проведения избирательной кампании устанавливается законом о выборах депутатов Государственного Собрания. Статус депутатов определяется федеральными законами, Конституцией и законами Республики Башкортостан. Государственное Собрание - Курултай является правомочным, если в его состав избрано не менее двух третей от установленного числа депутатов. На профессиональной постоянной основе исполнение полномочий законодательного органа обеспечивают 19 человек.

В соответствии с избирательным законодательством, половина депутатов избрана по одномандатным округам, половина – по партийным спискам. В выборах, состоявшихся в сентябре 2018 года, представители депутатские мандаты получили шести партий. По пропорциональной системе в Государственное Собрание – Курултай Республики Башкортостан шестого созыва прошли четыре партии — «Единая Россия», КПРФ, ЛДПР и «Справедливая Россия». По одномандатным округам депутатские мандаты получили представители еще двух партий: «Патриоты России» и Российская экологическая партия «Зеленые».

«Единая Россия» получила 79 мандатов, «КПРФ» – 15, ЛДПР – 7, «Справедливая Россия» – 5, «Патриоты России» – 2 мандата, Российская экологическая партия «Зеленые» – 1 мандат. Один мандат получил самовыдвиженец по Промышленному округу.

Деятельность Государственного Собрания – Курултая осуществляется в соответствии с Конституцией и законодательством Российской Федерации, Конституцией и законами Республики Башкортостан, его Регламентом и иными юридическими актами. Компетенция, порядок организации и деятельности парламента, его Президиума, постоянных комитетов, комиссий определяется Законом Республики Башкортостан «О Государственном Собрании – Курултае Республики Башкортостан» и законодательством Российской Федерации, Конституцией и законами Республики Башкортостан, иными юридическими актами. Компетенция, порядок организации и деятельности парламента, его Президиума, постоянных комитетов, комиссий определяется Законом Республики Башкортостан

Основной формой парламентской работы является заседание, которое являются правомочным, если на нем присутствует не менее 50 процентов от числа избранных депутатов. Государственное Собрание – Курултай, постоянные комитеты и комиссии вправе приглашать на заседания членов Правительства Республики Башкортостан, руководителей республиканских исполнительных органов, руководителей территориальных органов федеральных органов исполнительной власти.

Депутаты Государственного Собрания - Курултая, его постоянные комитеты являются субъектами права законодательной инициативы. Эти же права принадлежат Главе Республики Башкортостан, Правительству, Конституционному Суду, Центральной избирательной комиссии и Федерации профсоюзов, Прокурору Республики Башкортостан, представительным органам местного самоуправления.

В полномочия законодательного органа государственной власти республики относятся принятие Конституции, внесение в нее изменений, осуществление в пределах компетенции законодательного регулирования, толкование законов и ведение контроля за их исполнением.

Им утверждаются программы социально-экономического развития Башкортостана, бюджет и отчет о его реализации, заключение и расторжение договоров Республики Башкортостан, устанавливаются административно-территориальное устройство и порядок его изменения, налоги и сборы, порядок управления и распоряжения государственной собственностью.

Депутатами вносятся законопроекты и предложения об их разработке, выдвигаются кандидатуры на выборные должности Государственного Собрания - Курултая, ставится вопрос о приглашении должностных лиц органов государственной власти на заседания законодательного органа, его постоянных комитетов, комиссий, а также осуществляются иные действия, предусмотренные законодательством.

Государственным Собранием - Курултаем назначается дата, определяется порядок проведения республиканского референдума и выборов депутатов. Парламент участвует в решении кадровых вопросов, находящихся в сфере его компетенции, в установленном порядке взаимодействует с органами исполнительной и судебной власти, осуществляет право законодательной инициативы в Государственной Думе, проводит парламентские расследования и слушания.

Законодательный орган избирает Председателя Государственного Собрания - Курултая, который избирается большинством голосов от установленного числа депутатов тайным голосованием и осуществляет свои полномочия на профессиональной основе. В его полномочия входят ведение заседаний законодательного органа и Президиума, решение вопросов внутреннего распорядка их деятельности, осуществление общего руководства деятельностью Секретариата Государственного Собрания. Он назначает ответственный комитет и при необходимости комитет-соисполнитель за подготовку законопроекта, организует рассмотрение проекта закона, определяет сроки подготовки отзывов, предложений и замечаний. При наличии достаточных правовых оснований в установленном порядке возвращает законопроект субъекту права законодательной инициативы.

Председатель представляет Государственное Собрание в отношениях с федеральными органами государственной власти, органами государственной власти Республики Башкортостан и субъектов Российской Федерации, органами местного самоуправления, общественными объединениями, другими организациями и должностными лицами, направляет для подписания Главе Республики Башкортостан принятые законы.

Заместитель Председателя выполняет по поручению Председателя Государственного Собрания - Курултая отдельные полномочия и замещает его в случае отсутствия или невозможности осуществления им своих обязанностей.

Основными структурными подразделениями законодательного органа являются Президиум, постоянные комитеты и комиссии. Порядок их формирования и деятельности определяется Регламентом Государственного Собрания - Курултая Республики Башкортостан и положениями о постоянных его комитетах и комиссиях.

Президиум является постоянно действующим органом, подотчетным Государственному Собранию в своей деятельности. Он создается для подготовки и предварительного рассмотрения организационных вопросов, осуществления иных полномочий, предусмотренных законодательством. В его состав входят Председатель Государственного Собрания - Курултая, его заместитель, председатели постоянных комитетов.

Комитет ведет разработку и предварительное рассмотрение законопроектов по предметам ведения, составление по ним заключений. В установленном порядке участвует в парламентских слушаниях и расследованиях, осуществляет контроль за исполнением нормативных правовых актов, рассматривает организационные и иные вопросы, отнесенные к его ведению. Его деятельность регулируется председателем, который осуществляет свои полномочия на профессиональной основе.

В осуществлении функций Государственного Собрания - Курултая важную роль играют фракции и депутатские группы. Фракции формируются депутатами, избранными по партийным спискам. В них могут также входить лица, избранные по одномандатным избирательным округам. Депутаты, не вошедшие во фракции, образуют депутатские группы.

Работа фракций и депутатских групп, комитетов, комиссий и Президиума обеспечивает осуществление представительской и законодательской деятельности Государственного Собрания - Курултая, подготовку и принятие его решений. Они подразделяются на законы, постановления, обращения, заявления и декларации.

Установлен порядок работы и принятия решений Государственным Собранием - Курултаем. Конституция, законы Республики Башкортостан о внесении изменений и дополнений в Основной закон принимаются большинством не менее двух третей голосов от установленного числа депутатов. Закон Республики Башкортостан принимается простым большинством голосов от установленного числа депутатов, если иное не предусмотрено федеральным законом. Его проект рассматривается не менее чем в двух чтениях. Решение о его принятии либо отклонении оформляется постановлением Государственного Собрания - Курултая.

Постановления принимаются большинством голосов от числа избранных депутатов, если иное не предусмотрено федеральным законом. Решения о принятии обращений, заявлений, деклараций оформляются постановлениями Государственного Собрания – Курултая Республики Башкортостан.

\insertpicc{emil/upravlenie.jpg}{Местное управление}

